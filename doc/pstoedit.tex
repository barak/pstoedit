\documentclass[english,a4paper]{article}
\usepackage[latin1]{inputenc}
\usepackage{babel}

%% do we have the `fancyhdr' package?
\IfFileExists{fancyhdr.sty}{
\usepackage[fancyhdr]{latex2man}
}{
%% do we have the `fancyheadings' package?
\IfFileExists{fancyheadings.sty}{
\usepackage[fancy]{latex2man}
}{
\usepackage[nofancy]{latex2man}
\message{no fancyhdr or fancyheadings package present, discard it}
}}

\setVersionWord{Version}  %%% that's the default, no need to set it.
\setVersion{3.31}
\setDate{2001/12/30}    

\setlength{\emergencystretch}{1.5em}

\usepackage{url}
\let\URL\url \let\Email\url \let\File\url


\begin{document}

\begin{Name}{1}{pstoedit}{Dr. Wolfgang Glunz}{Conversion Tools}{PSTOEDIT}
  \Prog{pstoedit} - a tool converting PostScript and PDF files into various
  vector graphic formats
\end{Name}

\section{Synopsis}
%%%%%%%%%%%%%%%%%%

\subsection{From the command shell}

\Prog{pstoedit} \oOpt{-v -help}

\Prog{pstoedit} \oOpt{-adt}
		\oOptArg{-df}{~fontname}
		\oOpt{-dis}
        	\oOpt{-dt}
		\oOptArg{-flat}{~number}
		\oOptArg{-fontmap}{~file}
		\oOptArg{-include}{~includefile}
		\oOpt{-merge}
		\oOpt{-nb}
		\oOpt{-nfr}
		\oOpt{-nomaptoisolatin1}
		\oOpt{-nq}
		\oOptArg{-page}{~number}
		\oOptArg{-pagesize}{~string}
		\oOptArg{-psarg}{~string}
		\oOpt{-pti or -pta}
		\oOpt{-rgb}
		\oOptArg{-rotate}{~angle (0-360)}
		\oOptArg{-scale}{~factor}
		\oOpt{-sclip}
		\oOpt{-ssp}
		\oOpt{-split}
		\oOpt{-t2fontsast1}
		\oOptArg{-uchar}{~character}
		\oOpt{-v}
		\OptArg{-f}{~"format[:options]"} 
		[\Arg{inputfile} \oArg{outputfile}]
	       
\Prog{pstoedit} \oOptArg{-scale}{~factor}
		\OptArg{-f}{~"format[:options]"}
                \Opt{-bo} 
		\Arg{input-file} \oArg{output-file}  
		      
\subsection{From Gsview}

From within gsview pstoedit can be called via 
"\textbf{Edit | Convert to vector format}"


\subsection{From programs that support the ALDUS graphic import filter interface}
  
\Prog{pstoedit} can also be used as PostScript and PDF graphic import filter for several programs including
MS-Office 95/97,2000, PaintShop-Pro and PhotoLine. See 
\URL{http://www.pstoedit.net/importps/} for more 
details.

 
\section{Description}
%%%%%%%%%%%%%%%%%%%%%

\subsection{RELEASE LEVEL}

This manpage documents release \Version\ of \Prog{pstoedit}. 

\subsection{USE}

\Prog{pstoedit} converts PostScript and PDF files to various vector graphic
formats. The resulting files can be edited or imported into various drawing
packages. Type 

     \textbf{pstoedit -help} 
     
\noindent to get a list of supported output formats. Pstoedit comes with a
large set of format drivers integrated in the binary. Additional drivers can be
installed as plugins and are available via 
\URL{http://www.pstoedit.net/plugins/}. Just
copy the plugins to the same directory where the pstoedit binary is installed.
However, unless you also get a license key for the plugins, the additional
drivers will slightly distort the resulting graphics. See the documentation
provided with the plugins for further details.

\subsection{PRINCIPLE OF CONVERSION}

\Prog{pstoedit} works by redefining the two basic painting operators of
PostScript, \textbf{stroke} and \textbf{show} (bitmaps drawn by the image
operator are not supported by all backends.) After
redefining these operators, the PostScript or PDF file that needs to be
converted is processed by a PostScript interpreter, e.g., Ghostscript
(\Cmd{gs}{1}). You normally need to have a PostScript interpreter installed in
order to use this program. However, you can perform some "back end" processing
of prepared files by specifying the \Opt{-bo} option for debugging or limited
filtering. See "BACK END-SPECIFIC OPTIONS" below. 

The output that is written by the interpreter due to the redefinition of the
drawing operators is a sort of 'flat' PostScript file that contains only simple
operations like moveto, lineto, show, etc. You can look at this file using the
\Opt{-f debug} option. 

This output is read by end-processing functions of \Prog{pstoedit} and triggers
the drawing functions in the selected back end driver, or backend. 

\subsection{NOTES}

If you want to process PDF files directly, your PostScript interpreter must
provide this feature, as does Ghostscript. Aladdin Ghostscript 4.03 or later is
recommended for processing PDF (and PostScript Level 2) files.

\section{Options}
%%%%%%%%%%%%%%%%%

\begin{description}


\item[\Opt{-dt}] Draw text - Text is drawn as polygons. This might produce a large output file. This option is automatically
     switched on if the selected backend does not support this, e.g. 
     \Cmd{gnuplot}{1}.


\item[\Opt{-adt}] Automatic Draw text - This option turns on the -dt option selectively for fonts that seem to be no normal text fonts, e.g. Symbol..

\item[\Opt{-t2fontsast1}] Handle type 2 fonts same as type 1. Type 2 fonts sometimes occur as
embedded fonts within PDF files. In the default mode, text using such fonts is drawn as polygons
since pstoedit assumes that such a font is not available on the users machine. If this option
is set, pstoedit assumes that the internal encoding follows the same as for a standard font
and generates normal text output. This assumption may not be true in all cases. But it
is nearly impossible for pstoedit to verify this assumption - it would have to do a sort of OCR.

\item[\Opt{-pti or -pta}] Precision text - With \Opt{-pta}, each character of a text string is placed 
	separately. With \Opt{-pti}, this is done only in cases when there is a non zero inter-letter 
	spacing. Normally
     	a text string is drawn as it occurs in the input file. However, in some situations, this might
	produce wrongly positioned characters. This is due to limitiations in most backends of 
	pstoedit. They cannot represent text with arbitray inter-letter spacing which is easily
	possible in PDF and PostScript. The downside of "precision text" is a bigger file size and hard to edit
	text.

\item[\Opt{-nfr}] In normal mode pstoedit replaces bitmap fonts with a font as defined by the \Opt{-df} option. This is done, because most backends can't handle such fonts. This behavior can be 
switched off using the \Opt{-nfr} option but then it strongly depends on the application reading the the generated file whether the file is usable and correctly interpreted or not. Any problems are then out of control of pstoedit.


\item[\Opt{-dis}] Open a display during processing by Ghostscript. Some files
     only work correctly this way. 

\item[\OptArg{-psarg}{~string}] The string given with this option is passed
     directly to Ghostscript when Ghostscript is called to process the
     PostScript file for \Prog{pstoedit}. For example:     
     %\Opt{-psarg "-r300x300"} % latex2man bug (")
     \textbf{-psarg}\textbf{~"}\textbf{-r300x300}\textbf{"}
     This causes the resolution to be changed to
     300x300 dpi. (With older versions of GhostScript, changing the resolution
     this way has an effect only if \Opt{-display} is set.) 

\end{description} 

You can switch Ghostscript into PostScript Level 1 only mode by 
\Opt{-psarg "level1.ps"}. This can be useful for example if the PostScript file to be
converted uses some Level 2 specific custom color models that are not supported
by pstoedit. However, this requires that the PostScript program checks for the
PostScript level supported by the interpreter and "acts" accordingly. 

If you want to pass multiple options to Ghostscript you must can use multiple 
-psarg options \Opt{-psarg opt1} \Opt{-psarg opt2} \Opt{-psarg opt2}.
See the GhostScript manual for other possible options. 

\begin{description}

\item[\Opt{-merge}] Some output formats permit the representation of filled
     polygons with edges that are in a different color than the fill color.
     Since PostScript does not support this, drawing programs typically
     generate two objects (the outline and the filled polygon) into the
     PostScript output. \Prog{pstoedit} is able to recombine these, if they
     follow each other directly and you specify  \Opt{-merge}.
     
\item[\OptArg{-page}{~page number}]Select a single page from a multi page
     PostScript or PDF file. 

\item[\OptArg{-rotate}{~angle (0-360)}]Rotage image by angle.

\item[\Opt{-rgb}] Since version 3.30 pstoedit uses the CMYK colors internally. The -rgb option turns on the old behavior to use RGB values.


\item[\Opt{-split}] Create a new file for each page of the input. For this the
    output filename must contain a \%d which is replaced with the current page
    number. This option is automatically switched on for backends that don't
    support multiple pages within one file, e.g. fig or gnuplot. 

\item[\OptArg{-uchar}{~character}] Sometimes pstoedit cannot map a character
from the encoding used by the PostScript file to the font encoding of the target
format. In this case pstoedit replaces the input character by a special character
in order to show all the places that couldn't be mapped correctly. The default
for this is a "\#". Using the \Opt{-uchar} option it is possible to specify another character
to be used instead. If you want to use a space, use -uchar " ".

\item[\OptArg{-df}{~fontname}] Sometimes fonts embedded in a PostScript
    programs do not have a fontname. For example, this happens in PostScript
    files generated by \Cmd{dvips}{1}. In such a case \Prog{pstoedit} uses a
    replacement font. The default for this is Courier. Another font can be
    specified using the \Opt{-df} option. \Opt{-df Helvetica} causes all
    unnamed fonts to be replaced by Helvetica. 

\item[\OptArg{-include}{~name of a PostScript file to be included}] This
    options allows to specify an additional PostScript file that will be
    executed just before the normal input is read. This is helpful for
    including specific page settings or for disabling potentially unsafe
    PostScript operators, e.g., file, renamefile, or deletefile.

\item[\OptArg{-fontmap}{~name of font map file for pstoedit}] The font map is a
     simple text file containing lines in the following format: 

     \verb+document_font_name    target_font_name+ \\   
% bug in latex2html (second % char)!
     \verb+% lines beginning with + \verb+% are comments+ \\    
     \verb+% if a font name contains spaces, use+ \\
     \verb+% the "font name with spaces" notation.+
     
    Each font name found in the document is checked agains this mapping and if
    there is a corresponding entry, the new name is used for the output.

    If  the \Opt{-fontmap} option is not specified, \Prog{pstoedit}
    automatically looks for the file \emph{drivername}.fmp in the installation
    directory and uses that file as a default fontmap file if available. The
    installation directory is:
    
    \begin{itemize}
    
      \item Windows: The same directory where the \Prog{pstoedit} executable is
		    located 

      \item Unix: \\
	\verb+<+
	\emph{The directory where the pstoedit executably is located}
	\verb+>/../lib/+ 
	
    \end{itemize}

    The mpost.fmp in the misc directory of the pstoedit distibution is a sample
    map file with mappings from over 5000 PostScript font names to their \TeX
    equivalents. This is useful because MetaPost is frequently used with
    \TeX/\LaTeX\ and those programs don't use standard font names. This file and
    the MetaPost backend are provided by Scott Pakin
    (\Email{pakin@cs.uiuc.edu}). 

    Another example is wemf.fmp to be used under Windows. See the misc
    directory of the pstoedit distribution. 

\item[\OptArg{-f}{~format}] target output format recognized by
    \Prog{pstoedit}.  Since other format drivers can be loaded dynamically,
    type  \texttt{pstoedit -help} to get a full list of formats. See  "BACK
    END-SPECIFIC OPTIONS" below for an explanation of the \oArg{:options} to
    \Opt{-f} format. 

\item[\OptArg{-scale}{~factor}] scale by the specified factor. (Currently used with
     \Opt{-f tgif} backend only.) 

\item[\Opt{-ssp}] simulate sub paths \\ 
    Several backend don't support PostScript pathes containing sub pathes, i.e.
    pathes with intermediate movetos. In the normal case, each subpath is
    treated as an independent path for such backends. This can lead to bad
    looking results. The most common case where this happens is if you use the
    \Opt{-dt} option and show some text with letters like e, o, or b, i.e.
    letter that have a "hole". When the \Opt{-ssp} option is set, pstoedit
    tries to eliminate these problems. However, this option is CPU time
    intensive! 

\item[\Opt{-sclip}] simulate clipping \\
    Most backends of pstoedit don't have native support for clipping. For that
    \Prog{pstoedit} offers an option to perform the clipping of the graphics
    directly without passing the clippath to the backends. However, this
    results in curves being replaces by a lot of line segments and thus larger
    output files. So use this option only if your output looks different from
    the input due to clipping. In addition, this "simulated clipping" is not
    exactly the same as defined in PostScript. There might be lines drawn at
    the double size. Also clipping of text is not supported unless you also use
    the \Opt{-dt} option. 

\item[\OptArg{-pagesize}{string}] set page size for output medium \\
    This option sets the page size for the output medium. Currently this
    is just used by the libplot backend, but might be used by other 
    backends in future. The page size is specified in terms of the usual
    page size names, e.g. letter or a4.

\item[\Opt{-bo}] You can run backend processing only (without the PostScript
    interpreter frontend) by first running \textbf{pstoedit} \Opt{-f dump}
    \Arg{infile} \Arg{dumpfile} and then running \textbf{pstoedit}
    \OptArg{-f}{~format}  \Opt{-bo} \Arg{dumpfile} \Arg{outfile}. 

\item[\OptArg{-flat}{~number}] If the backend does not support curves in the way
    PostScript does or if the \Opt{-nc} option is specified, all curves are
    approximated by lines. Using the \Opt{-flat} option one can control this
    approximation. This parameter is directly converted to a PostScript
    \textbf{setflat} command. Higher numbers, e.g. 10 give rougher, lower
    numbers, e.g. 0.1 finer approximations.  

\item[\Opt{-nb}] Since version 3.10 \Prog{pstoedit} uses the
    \texttt{-dDELAYBIND} option when calling GhostScript. Previously the
    \texttt{-dNOBIND} option was used instead but that sometimes caused
    problems if a user's PostScript file overloaded standard PostScript
    operator with totally new semantic, e.g. lt for lineto. Using \Opt{-nb} the
    old style can be activated again in case the \texttt{-dDELAYBIND} gives
    different results as before. In such a case please also contact the
    author. 

\item[\Opt{-nc}] no curves \\
    Normally pstoedit tries to keep curves from the input and transfers them to
    the output if the output format supports curves. If the backend does not
    support curves, then pstoedit replaces curves by a series of lines (see
    also \Opt{-flat} option). However, in some cases the user might wish to
    have this behavior also for backends that originally support curves. This
    can be forced via the \Opt{-nc} option. 
    
\item[\Opt{-nq}] No exit from the PostScript interpreter. Normally Ghostscript
    exits after processing the pstoedit input-file. For debugging it can be
    useful to avoid this. If you do, you will have to type quit at the
    \verb+GS>+ prompt to exit from Ghostscript.    

\item[\Opt{-v}] Switch on verbose mode. Some additional information is shown
     during processing. 

\item[\Opt{-nomaptoisolatin1}] Normally \Prog{pstoedit} maps all character
    codes to the ones defined by the ISO-Latin1 encoding. If you specify
    \Opt{-nomaptoisolatin1} then the encoding from the input PostScript is
    passed unchanged to the output. 

\item[\Arg{input-file}] input file.  If a "-" is given, standard input is used.

\item[\Arg{output-file}] output file. If no output file or "-" is given as argument,
    \Prog{pstoedit} writes the result to standard output. 

\end{description}



If neither an input nor an output file is given as argument, pstoedit works as filter reading from standard input and
writing to standard output. 


\section{BACK END-SPECIFIC OPTIONS}

\Prog{pstoedit} allows you to pass individual options to a backend. This is done by
appending all options to the format specified after the \Opt{-f} option. The format
specifier and its options must be separated by a colon (:). If more than one
option needs to be passed to the backend, the whole argument to \Opt{-f} must be
enclosed within double-quote characters, thus:

\OptArg{-f}{~"format[:option option ...]"}
%\textbf{-f}\textbf{~"}\textbf{format:option option ...}\textbf{"} % latex2man bug (")

To see which options are supported by a specific format, type:
     \textbf{pstoedit -f format:-help} 

The following description is it not up to date at the moment. Sorry! Please use the above command
to get a current list of options supported by the specific format.

Currently \emph{met}, \emph{java}, \emph{dxf}, \emph{pic}, \emph{fig},
\emph{metapost}, \emph{LaTeX2e}, \emph{mif}, \emph{emf}, and \emph{wmf} are the only drivers accepting specific options.
Other options may be asserted through environment variables. See "ENVIRONMENT
VARIABLES" below. 

\noindent
The \emph{wmf} and the \emph{emf} driver supports the following backend specific options:

\begin{description}

\item[\Opt{-f wmf:m}] Maps all fonts in the document to Arial (should be 
	available on every Windows installation)
\item[\Opt{-f wmf:n}] Emulate narrow fonts by shrinking fonts horizontally 
	(sometimes does not look that good, but it's the only chance, when 
	requested font weight is not available. And this is quite common for 
	off-the-shelf Windows installations)
\item[\Opt{-f wmf:b}] DON'T draw two white border pixel (upper left and lower 
	right corner). They are normally drawn to keep content always within 
	bounding box (is sometimes clipped otherwise, i.e. Windows doesn't 
	respect pen thickness or rotated text extents).
	This could be done more smarter for EMF, have to figure out...

\end{description}

The \emph{java} backend allows to specify the class name of the class that is
generated by pstoedit. The default is PSJava. You can change this using 

\begin{description}

  \item[\OptArg{-f java:}{anothername}] 

\end{description}


The \emph{dxf} backend accepts the option \Opt{-lines} which forces all
polygons and lines to be represented as LINEs in the generated DXF file. The
default is to use POLYLINEs. 

\noindent Example: 
  %\Opt{-f~"dxf:-lines"}
  \textbf{-f}\textbf{~"}\textbf{dxf:-lines}\textbf{"} % latex2man bug (")


\noindent The \emph{met} backend allows the following single character options (without a
leading -) 

\begin{description}

  \item[\Opt{p}] Draw no geometric linewidths, all lines have a width of zero.
  \item[\Opt{l}] No filling of polygon interiors.
  \item[\Opt{c}] No colors, just greyscales.
  \item[\Opt{t}] Omit all text.
  \item[\Opt{g}] Omit all graphics.
  \item[\Opt{v}] Put verbose output to \File{STDERR}. 

\end{description}


Example: 
  %\Opt{-f "met:wlc"}
  \textbf{-f}\textbf{~"}\textbf{met:lc}\textbf{"} % latex2man bug


\noindent The \emph{pic} backend accepts the options:

\begin{description}

  \item[\Opt{-troff}]
  \item[\Opt{-groff}] 
     which forces output to be compatible with troff and groff,
     respectively. Groff mode is default, troff mode severely limiting
     the choice of supported text fonts.

     BUG: these options really does not belong in a backend

  \item[\Opt{-keep}]
     makes the pic backend emit the full font name of fonts that does
     not map to built-in groff fonts.

  \item[\Opt{-text}]
     makes the pic backend attempt to recognize running text, and
     treat  it accordingly.

  \item[\Opt{-landscape}]
  \item[\Opt{-portrait}]
     to compensate for the postscript orientation. Portrait mode is
     default.

     Example:
     %\Opt{-f "pic:-troff -text -landscape"}
     \textbf{-f}\textbf{~"}\textbf{pic:-troff -text -landscape}\textbf{"} 
     % latex2man bug
     
\end{description}

\noindent The \emph{fig} driver accepts the following options:

\begin{description}

  \item[\Opt{-startdepth}] 
    Fig knows about 999 layers (0 is the topmost, 999 the backmost). Per
    default, pstoedit starts with layer 999 and then places all subsequent
    objects on lower layers (on top of the previous objects). This can result
    in problems if you want to put something "below" all the objects that
    were created by pstoedit. In such a case you should define a lower number
    to leave some space behind.
     
  \item[\Opt{-depth}] 
    depth in inches. Sets the paper width to the specified size in inches.
     
\end{description}

The \emph{ps2ai} driver provides an option to select to old AI-88 format
instead of the default AI-3

\begin{description}
  
  \item[\Opt{-88}] selects the AI-88 format.
  
     Example: \Opt{-f ps2ai:-88}
     
\end{description}

The \emph{tgif} driver provides option to control the conversion of text
strings into hyperlink attributes.

\begin{description}

  \item[\Opt{-ta}] 
     enables the conversion of text into boxes with hyperlink attributes. More
     options allowing finer control about this new feature will follow in
     future versions.
     
\end{description}

\noindent
The \emph{tk} driver supports the following backend specific options:

\begin{description}

\item[\Opt{-f tk:I}] Disables ImPress specific formatting.  Only canvas
    objects will be output.
\item[\Opt{-f tk:N tagname}] Adds a specific tag to all objects.  If ImPress
    formatting is enabled, the items will be grouped.
\item[\Opt{-f tk:n tagname}] Deprecated option.  Behaves like N.
\item[\Opt{-f tk:R}] If ImPress formatting is enabled, swap the Width and
    Height associated with the pagesize.

\end{description}
 
The \emph{mif} backend allows the following options:

\begin{description}

  \item[\Opt{-f mif:-nopage}] Generates an anchored frame instead of a full page.
This is useful, if you want to insert a figure into an existing document.

  \item[\Opt{-f mif:-imagesaspng}] Bitmap images are written as PNG files instead
of EPS. This is still experimental but should work for non rotated bitmaps.

\end{description}

The GNU libplot driver (\emph{gmfa}, \emph{gmfb}, \emph{plot}) provides a huge
set of options. All these are described in the header of the drvlplot.cpp file.



\subsection{NOTES}

\begin{description}

  \item[autotrace:]

	pstoedit cooperates with autotrace. Autotrace can now produce a dump file 
	for further processing by pstoedit using the \Opt{-bo} (backend only) option. 
	Autotrace is a program written by a group around Martin Weber and can be 
	found at \URL{http://sourceforge.net/projects/autotrace/}.

  \item[Ps2ai:]

    The ps2ai backend is not a native pstoedit backend. It does not use the
    pstoedit postcript flattener, instead it uses the PostScript program
    ps2ai.ps which is installed in the GhostScript distribution directory. It
    is included  to provide the same "look-and-feel" for the conversion to AI.
    The additional benefit is that this conversion is now available also via
    the "convert-to-vector" menu of Gsview. However, lot's of files don't
    convert nicely or at all using ps2ai.ps. So a native pstoedit driver would
    be much better. Anyone out there to take this ? The AI format is usable for
    example by Mayura Draw (\URL{http://www.mayura.com}). Also a driver to the
    Mayura native format would be nice. 

    If you have a version of GhostScript older than 5.60, then 
    you have to apply the following simple patch to the
    \File{ps2ai.ps} file in order to make this driver work.
    This patch is already included in newer versions of GhostScript. 

    After the line "\verb+/vers {2.13} def+" insert:
    
    \verb+/cdef { 1 index where { pop pop pop } { def } ifelse } def+
    
    Replace the lines:
    
    "\verb+/jout false def+"  
    
    with 
    
    "\verb+/jout false cdef+"  
    
    (notice the cdef instead of def)
    
    "\verb+/joutput (ps2ai.out.aips) def+" 
    
    with 
    
    "\verb+/joutput (ps2ai.out.aips) cdef+"
    
    "\verb+/joutln false def+" 
    
    with 
    
    "\verb+/joutln false cdef+"

    and the line
    
    "\verb+/jtxt3 true def+" 
    
    with 
    
    "\verb+/jtxt3 true cdef+"

    Note: If you already patched \emph{ps2ai} for pstoedit version 3.02. you
    have to change to the patch above. This version is different but it better
    fits the ideas of Peter L. Deutsch. Sorry for the confusion, but this way
    chances are better that this version will go into the GhostScript
    distribution.


  \item[MetaPost:]

    Note that, as far as Scott knows, MetaPost does not support PostScript's
    eofill. My backend just converts eofill to fill, and issues a warning if
    verbose is set. Fortunately, very few PostScript programs rely on the
    even-odd fill rule, even though many specify it.

    For more on MetaPost see: 
    
    \URL{http://cm.bell-labs.com/who/hobby/MetaPost.html}

  \item[LaTeX2e:] \ \\

    \begin{itemize}
   \item LaTeX2e's picture environment is not very powerful.  As a result, many
     elementary PostScript constructs are ignored -- fills, line
     thicknesses (besides "thick" and "thin"), and dash patterns, to name a
     few.  Furthermore, complex pictures may overrun TeX's memory capacity.

   \item Some PostScript constructs are not supported directly by "picture",
     but can be handled by external packages.  If a figure uses color, the
     top-level document will need to do a \verb+"\usepackage{color}"+.  And if a
     figure contains rotated text, the top-level document will need to do a
     \verb+"\usepackage{rotating}"+.

   \item All lengths, coordinates, and font sizes output by the backend are in
     terms of \verb+\unitlength+, so scaling a figure is simply a matter of doing
     a \verb+"\setlength{\unitlength}{...}"+.

   \item The backend currently supports one backend-specific option,
     "integers", which rounds all lengths, coordinates, and font sizes to
     the nearest integer.  This makes hand-editing the picture a little
     nicer.

   \item Why is this backend useful?  One answer is portability; any LaTeX2e
     system can handle the picture environment, even if it can't handle
     PostScript graphics.  (pdfLaTeX comes to mind here.)  A second answer
     is that pictures can be edited easily to contain any arbitrary LaTeX2e
     code.  For instance, the text in a figure can be modified to contain
     complex mathematics, non-Latin alphabets, bibliographic citations, or
     -- the real reason Scott wrote the LaTeX2e backend -- hyperlinks to the
     surrounding document (with help from the hyperref package).
   \end{itemize}


  \item[creating a new backend:]

    To implement a new backend you can start from \File{drvsampl.cpp} and
    \File{drvsampl.h}. See also comments in \File{drvbase.h} and
    \File{drvfuncs.h} for an explanation of methods that should be implemented
    for a new backend.

\end{description}

\subsection{ENVIRONMENT VARIABLES}

A default PostScript interpreter to be called by pstoedit is specified at
compile time. You can overwrite the default by setting the GS environment
variable to the name of a suitable PostScript interpreter.

You can check which name of a PostScript interpreter was compiled into
pstoedit using: \textbf{pstoedit} \Opt{-help -v}.

See the GhostScript manual for descriptions of environment variables used by
Ghostscript most importantly \verb+GS_FONTPATH+ and \verb+GS_LIB+; other
environment variables also affect output to display, print, and additional
filtering and processing. See the related documentation.

\Prog{pstoedit} allocates temporary files using the function \Cmd{tempnam}{3}.
Thus the location for temporary files might be controllable by other
environment variables used by this function. See the \Cmd{tempnam}{3} manpage
for descriptions of environment variables used. On UNIX like system this is
probably the \verb+TMPDIR+ variable, on DOS/WINDOWS either \verb+TMP+ or
\verb+TEMP+.


\subsection{SYSTEM SPECIFIC NOTES}

\begin{description}

  \item[DOS/WINDOWS]

    pstoedit compiled with MS-Visual C++ or Borland C++ runs under 32-bit
    only. It might run under WIN32s, but certainly does not run under plain
    16-bit DOS.

    \Prog{pstoedit} works best if you installed at least version 5.50 of
    GhostScript and version 2.72 of gsview. Using older version of
    GhostScript is possible but requires the setting of some environment
    variables.

\end{description}

\subsection{TROUBLE SHOOTING}

If you have problems with \Prog{pstoedit} first try whether Ghostscript
successfully displays your file. If yes try 
\textbf{pstoedit} \Opt{-f ps} \Arg{infile.ps} \Arg{testfile.ps} 
and check whether \Arg{testfile.ps} still displays correctly using
Ghostscript. If this file doesn't look correctly then there seems to be a
problem with \Prog{pstoedit}'s PostScript frontend. If this file looks good
but the output for a specific format is wrong, the problem is probably in
the backend for the specific format. In either case send bug fixes and
reports to the author.

A common problem with PostScript files is that the PostScript file redefines
one of the standard PostScript operators inconsistently. There is no effect
of this if you just print the file since the original PostScript "program"
uses these new operator in the new meaning and does not use the original
ones anymoew. However, when run under the control of pstoedit, these
operators are expected to work with the original semantics.

So far I've seen redefinitions for:

\begin{itemize}

   \item lt - "less-then" to mean "draw a line to"
   \item string - "create a string object" to mean "draw a string"
   \item length - "get the length of e.g. a string" to a "float constant"
   
\end{itemize}

I've included work-arounds for the ones mentioned above, but some others
could show up in addition to those.


\subsection{RESTRICTIONS}

Non-standard fonts (e.g. \TeX bitmap fonts) are mapped to a default font which
can be changed using the \Opt{-df} option. \Prog{pstoedit} chooses the size of
the replacement font such that the width of the string in the original font is
the same as in the replacement font. This is done for each text fragment
displayed. Special character encoding support is limited in this case. If a
character cannot be mapped into the target format, pstoedit displays a '\#'
instead. See also the -uchar option.

pstoedit supports bitmap graphics only for some backends.

The Gnuplot backend and the 3D backends (rpl, lwo, rib) do not support text.

Generally, pstoedit does not support clipping. You can try to use the
\Opt{-sclip} option to simulate clipping. However, this doesn't work in all cases
as expected.

Special note about the Java backends (java1 and java2)

The java backends generate a java source file that needs other files in
order to be compiled and usable. These other files are Java classes (one
applet and support classes) that allow to step through the individual pages
of a converted PostScript document. This applet can easily be activated from
a html-document. See the \File{java/java1/readme_java1.txt} or 
\File{java/java2/readme_java2.htm} file for more details.

\subsection{FAQs}

Why do letters like O or B get strange if converted to tgif/xfig
using the \Opt{-dt} option?

This is because most backends don't support composite paths with
intermediate gaps (moveto's) and second don't support very well the (eo)fill
operators of PostScript (winding rule). For such objects \Prog{pstoedit} breaks
them into smaller objects whenever such a gap is found. This results in the
"hole" beeing filled with black color instead of beeing transparent. Since
version 3.11 you can try the \Opt{-ssp} option in combination with the xfig
backend.


Why does pstoedit produce ugly results from PostScript files generated
by dvips?

TeX documents usually use bitmap fonts. Such fonts cannot be used as native
font in other format. So pstoedit replaces the TeX font with another native
font. Of course, the replacement font will in most cases produce another
look, especially if mathematical symbols are used.

\subsection{NOTICES}

\subsection{AUTHOR}

Wolfgang Glunz, \Email{wglunz@pstoedit.net}


\subsection{CANONICAL ARCHIVE SITE}

\URL{http://www.pstoedit.net/pstoedit/}

At this site you also find more information about \Prog{pstoedit} and related
programs and hints how to subscribe to a mailing list in order to get informed
about new releases and bug-fixes.


\subsection{ACKNOWLEDGEMENTS}

\begin{itemize}\setlength{\itemsep}{0cm}

  \item Klaus Steinberger \Email{Klaus.Steinberger@physik.uni-muenchen.de}
     wrote the initial version of this manpage.

  \item Lar Kaufman \Email{lark@walden.com} revised the increasingly complex
     command syntax diagrams and updated the structure and content of this
     manpage following release 2.5. (\URL{http://www.walden.com/~lark/})

  \item David B. Rosen \Email{rosen@unr.edu} provided ideas and some PostScript
     code from his ps2aplot program.

  \item Ian MacPhedran \Email{Ian_MacPhedran@engr.USask.CA} provided the xfig
     backend.

  \item Carsten Hammer \Email{chammer@hermes.hrz.uni-bielefeld.de} provided the
     gnuplot backend and the initial DXF backend.

  \item Christoph Jaeschke provided the OS/2 metafile (MET) backend. 
  Thomas Hoffmann \Email{thoffman@zappa.sax.de} did some further updates on the OS/2 part.

  \item Jens Weber \Email{rz47b7@PostAG.DE} provided the Windows metafile (WMF)
     backend, and a graphical user interface (GUI).

  \item G. Edward Johnson \Email{lorax@nist.gov} provided the CGM Draw library
     used in the CGM backend.

  \item Gerhard Kircher \Email{kircher@edvz.tuwien.ac.at} provided some bug
     fixes.

  \item Bill Cheng \Email{william@cs.columbia.edu} provided help with the tgif
     format and some changes to tgif to make the backend easier to implement.
     URL:\URL{http://www.cs.columbia.edu/~william}

  \item Reini Urban \Email{rurban@sbox.tu-graz.ac.at} provided input for the
     extended DXF backend.(\URL{http://xarch.tu-graz.ac.at/autocad/})

  \item Glenn M. Lewis \Email{glenn@gmlewis.com} provided RenderMan (RIB),
     Real3D (RPL), and LightWave 3D (LWO) backends.
     (\URL{http://www.gmlewis.com/})

  \item Piet van Oostrum \Email{piet@cs.ruu.nl} made several bug fixes.
  
  \item Lutz Vieweg \Email{lkv@mania.robin.de} provided several bug fixes and
     suggestions for improvements.
     
  \item Derek B. Noonburg \Email{derekn@vw.ece.cmu.edu} and Rainer Dorsch
     \Email{rd@berlepsch.wohnheim.uni-ulm.de} isolated and resolved a
     Linux-specific core dump problem.

  \item Rob Warner \Email{rcw2@ukc.ac.uk} made pstoedit compile under RiscOS.
  
  \item Patrick Gosling \Email{jpmg@eng.cam.ac.uk} made some suggestions
     regarding the usage of pstoedit in Ghostscript's SAFER mode.

  \item Scott Pakin \Email{pakin@cs.uiuc.edu} for the Idraw backend and the 
	autoconf support.

  \item Peter Katzmann \Email{p.katzmann@thiesen.com} for the HPGL backend.
  
  \item Chris Cox \Email{ccox@airmail.net} contributed the Tcl/Tk backend.
  
  \item Thorsten Behrens \Email{Thorsten_Behrens@public.uni-hamburg.de} and
     Bjoern Petersen for reworking the WMF backend.

  \item Leszek Piotrowicz \Email{leszek@sopot.rodan.pl} implemented the image
     support for the xfig driver and a JAVA based GUI.

  \item Egil Kvaleberg \Email{egil@kvaleberg.no} contributed the pic backend.
  
  \item Kai-Uwe Sattler \Email{kus@iti.cs.uni-magdeburg.de} implemented the
     backend for Kontour.

  \item Scott Pakin,  pakin@cs.uiuc.edu   provided the MetaPost and LaTeX2e backend.
  
  \item Burkhard Plaum (\Email{plaum@IPF.Uni-Stuttgart.de}) added support for
     complex filled paths for the xfig backend.

  \item Bernhard Herzog (\Email{herzog@online.de}) contributed the backend for
     sketch ( \URL{http://sketch.sourceforge.net/} )

  \item Rolf Niepraschk (\Email{niepraschk@ptb.de}) converted the HTML man page
     to LaTeX. This allows to generate the UNIX style and the HTML manual from this
     base format.

  \item Several others sent smaller bug fixed and bug reports. Sorry if I don't
     mention them all here.

  \item Gisbert W. Selke (\Email{gisbert@tapirsoft.de}) for the Java 2 backend.
     
  \item Robert S. Maier (\Email{rsm@math.arizona.edu}) for many improvements on
	the libplot backend and for libplot itself.
  \item The authors of pstotext (\Email{mcjones@pa.dec.com} and \Email{birrell@pa.dec.com}) 
	for giving me the permission to use their simple PostScript code for 
	performing rotation.
  \item  Daniel Gehriger \Email{gehriger@linkcad.com} for his help concerning the handling of Splines in the DXF format. 
  \item Allen Barnett \Email{libemf@lignumcomputing.com} for his work on the libEMF which allows to create WMF/EMF files under *nix systems.
  \item Dave \Email{dave@opaque.net} for providing the libming which is a multiplatform library for generating SWF files.
  \item But most important: Peter Deutsch \Email{ghost@aladdin.com} and Russell
     Lang \Email{gsview@ghostgum.com.au} for their help and answers regarding
     GhostScript and gsview.

\end{itemize}

\subsection{LEGAL NOTICES}

Trademarks mentioned are the property of their respective owners.

Some code incorporated in the pstoedit package is subject to copyright or
other intellectual property rights or restrictions including attribution
rights. See the notes in individual files.

\Prog{pstoedit} is controlled under the Free Software Foundation GNU Public
License (GPL). However, this does not apply to importps and the additional
plugins.

Aladdin Ghostscript is a redistributable software package with copyright
restrictions controlled by Aladdin Software.

\Prog{pstoedit} has no other relation to Ghostscript besides calling it in a
subprocess.

The authors, contributors, and distributors of pstoedit are not responsible
for its use for any purpose, or for the results generated thereby.

Restrictions such as the foregoing may apply in other countries according to
international conventions and agreements.


\LatexManEnd

\end{document}
 
 
