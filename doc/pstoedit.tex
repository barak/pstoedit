\documentclass[english,a4paper]{article}
\usepackage[latin1]{inputenc}
\usepackage{babel}

%% do we have the fancyhdr package?
\IfFileExists{fancyhdr.sty}{
\usepackage[fancyhdr]{latex2man}
}{
%% do we have the fancyheadings package?
\IfFileExists{fancyheadings.sty}{
\usepackage[fancy]{latex2man}
}{
\usepackage[nofancy]{latex2man}
\message{no fancyhdr or fancyheadings package present, discard it}
}}

\setVersionWord{Version}  %%% that's the default, no need to set it.
\setVersion{3.61}
\setDate{August 2012}    

\setlength{\emergencystretch}{1.5em}

\usepackage{url}
\let\URL\url \let\Email\url \let\File\url

\begin{document}

\begin{Name}{1}{pstoedit}{Dr. Wolfgang Glunz}{Conversion Tools}{PSTOEDIT}
  \Prog{pstoedit} - a tool converting PostScript and PDF files into various
  vector graphic formats
\end{Name}

\section{Synopsis}
%%%%%%%%%%%%%%%%%%

\subsection{From the command shell}

\Prog{pstoedit} \oOpt{-v -help} 
\\  

\Prog{pstoedit} 
\oOptArg{-include}{~name of a PostScript file to be included} 
\oOptArg{-df}{~font name} 
\oOpt{-nomaptoisolatin1} 
\oOpt{-dis} 
\oOptArg{-pngimage}{~filename - for debugging purpose mainly. Write result of processing also to a PNG file.} 
\oOpt{-q} 
\oOpt{-nq} 
\oOpt{-nc} 
\oOpt{-nsp} 
\oOpt{-mergelines} 
\oOpt{-filledrecttostroke} 
\oOpt{-mergetext} 
\oOpt{-dt} 
\oOpt{-adt} 
\oOpt{-ndt} 
\oOpt{-dgbm} 
\oOpt{-correctdefinefont} 
\oOpt{-pti} 
\oOpt{-pta} 
\oOptArg{-xscale}{~number} 
\oOptArg{-yscale}{~number} 
\oOptArg{-xshift}{~number} 
\oOptArg{-yshift}{~number} 
\oOpt{-centered} 
\oOptArg{-minlinewidth}{~number} 
\oOpt{-split} 
\oOpt{-v} 
\oOpt{-usebbfrominput} 
\oOpt{-ssp} 
\oOptArg{-uchar}{~character} 
\oOpt{-nb} 
\oOptArg{-page}{~page number} 
\oOptArg{-flat}{~flatness factor} 
\oOpt{-sclip} 
\oOpt{-ups} 
\oOpt{-rgb} 
\oOpt{-useagl} 
\oOpt{-noclip} 
\oOpt{-t2fontsast1} 
\oOpt{-keep} 
\oOpt{-gstest} 
\oOpt{-nfr} 
\oOpt{-glyphs} 
\oOpt{-useoldnormalization} 
\oOptArg{-rotate}{~angle (0-360)} 
\oOptArg{-fontmap}{~name of font map file for pstoedit} 
\oOptArg{-pagesize}{~page format} 
\oOpt{-help} 
\oOptArg{-gs}{~path to the ghostscript executable/DLL } 
\oOpt{-bo} 
\oOptArg{-psarg}{~argument string} 
\oOptArg{-pslanguagelevel}{~PostScript Language Level to be used 1,2, or 3} 
\OptArg{-f}{~"format\Lbr:options\Rbr"} 
\oOptArg{-gsregbase}{~GhostScript base registry path} 
[ inputfile [outputfile] ] 

\subsection{From Gsview}

Pstoedit can be called from within gsview via 
"\textbf{Edit | Convert to vector format}"

\subsection{From programs that support the ALDUS graphic import filter interface}
  
\Prog{pstoedit} can also be used as PostScript and PDF graphic import filter for several programs including
MS-Office,  PaintShop-Pro and PhotoLine. See 
\URL{http://www.pstoedit.net/importps/} for more 
details.

 
\section{Description}
%%%%%%%%%%%%%%%%%%%%%

\subsection{RELEASE LEVEL}

This manpage documents release \Version\ of \Prog{pstoedit}. 

\subsection{USE}

\Prog{pstoedit} converts PostScript and PDF files to various vector graphic
formats. The resulting files can be edited or imported into various drawing
packages. Type 

     \textbf{pstoedit -help} 
     
\noindent to get a list of supported output formats. Pstoedit comes with a
large set of format drivers integrated in the binary. Additional drivers can be
installed as plugins and are available via 
\URL{http://www.pstoedit.net/plugins/}. 
Just copy the plugins to the same directory where the pstoedit binary is installed or - under Unix like systems only - alternatively into the lib directory parallel to the bin directory where pstoedit is installed.

However, unless you also get a license key for the plugins, the additional
drivers will slightly distort the resulting graphics. See the documentation
provided with the plugins for further details.

\subsection{PRINCIPLE OF CONVERSION}

\Prog{pstoedit} works by redefining the some basic painting operators of
PostScript, e.g. \textbf{stroke} or \textbf{show} (bitmaps drawn by the image
operator are not supported by all output formats.) After
redefining these operators, the PostScript or PDF file that needs to be
converted is processed by a PostScript interpreter, e.g., Ghostscript
(\Cmd{gs}{1}). You normally need to have a PostScript interpreter installed in
order to use this program. However, you can perform some "back end only" processing
of files following the conventions of the pstoedit intermediate formate by specifying the \Opt{-bo} option. See "Available formats and their specific options" below. 

The output that is written by the interpreter due to the redefinition of the
drawing operators is a sort of 'flat' PostScript file that contains only simple
operations like moveto, lineto, show, etc. You can look at this file using the
\Opt{-f debug} option. 

This output is read by end-processing functions of \Prog{pstoedit} and triggers
the drawing functions in the selected output format driver sometime called also "backend". 

\subsection{NOTES}

If you want to process PDF files directly, your PostScript interpreter must
provide this feature, as does Ghostscript. Aladdin Ghostscript is
recommended for processing PDF and PostScript files.

\section{Options}

\subsection{General options}
\begin{description}
\item[\oOptArg{-include}{~name of a PostScript file to be included}] 
This options allows to specify an additional PostScript file that will be executed just before the normal input is read. This is helpful for including specific page settings or for disabling potentially unsafe PostScript operators, e.g., file, renamefile, or deletefile. 


\item[\oOptArg{-xscale}{~number}] 
scale by a factor in x-direction


\item[\oOptArg{-yscale}{~number}] 
scale by a factor in y-direction


\item[\oOptArg{-xshift}{~number}] 
shift image in x-direction


\item[\oOptArg{-yshift}{~number}] 
shift image in y-direction


\item[\oOpt{-centered}] 
center image before scaling or shifting


\item[\oOptArg{-minlinewidth}{~number}] 
minimal line width. All lines thinner than this will be drawn in this line width - especially zero-width lines


\item[\oOpt{-split}] 
Create a new file for each page of the input. For this the output filename must contain a \%d which is replaced with the current page number. This option is automatically switched on for output formats that don't support multiple pages within one file, e.g. fig or gnuplot. 


\item[\oOpt{-usebbfrominput}] 
If specified, pstoedit uses the BoundingBox as is (hopefully) found in the input file instead of one that is calculated by its own. 


\item[\oOptArg{-page}{~page number}] 
Select a single page from a multi page PostScript or PDF file. 


\item[\oOpt{-rgb}] 
Since version 3.30 pstoedit uses the CMYK colors internally. The -rgb option turns on the old behavior to use RGB values.


\item[\oOpt{-useagl}] 
use Adobe Glyph List instead of the IsoLatin1 table (this is experimental)


\item[\oOpt{-noclip}] 
don't use clipping (relevant only if output format supports clipping at all)


\item[\oOptArg{-rotate}{~angle (0-360)}] 
Rotage image by angle.


\item[\oOptArg{-pagesize}{~page format}] 
set page size for output medium.  
This option sets the page size for the output medium. Currently this is just used by the libplot output format driver, but might be used by other  output format drivers in future. The page size is specified in terms of the usual page size names, e.g. letter or a4. 


\item[\oOpt{-help}] 
show the help information


\item[\oOptArg{-gs}{~path to the ghostscript executable/DLL }] 
tells pstoedit which ghostscript executable/DLL to use - overwrites the internal search heuristic


\item[\oOpt{-bo}] 
You can run backend processing only (without the PostScript interpreter frontend) by first running \textbf{pstoedit} \Opt{-f dump} \Arg{infile} \Arg{dumpfile} and then running \textbf{pstoedit} \OptArg{-f}{~format}  \Opt{-bo} \Arg{dumpfile} \Arg{outfile}. 


\item[\oOptArg{-psarg}{~argument string}] 
The string given with this option is passed directly to Ghostscript when Ghostscript is called to process the PostScript file for \Prog{pstoedit}. For example:      \textbf{-psarg}\textbf{~"}\textbf{-r300x300}\textbf{"}. This causes the resolution to be changed to 300x300 dpi. (With older versions of GhostScript, changing the resolution this way has an effect only if the \Opt{-dis} option is given.)  If you want to pass multiple options to Ghostscript you can use multiple  -psarg options \Opt{-psarg opt1} \Opt{-psarg opt2} \Opt{-psarg opt2}. See the GhostScript manual for other possible options. 


\item[\oOptArg{-pslanguagelevel}{~PostScript Language Level to be used 1,2, or 3}] 
PostScript Language Level to be used 1,2, or 3 You can switch Ghostscript into PostScript Level 1 only mode by  \Opt{-pslanguagelevel 1}. This can be useful for example if the PostScript file to be converted uses some Level 2 specific custom color models that are not supported by pstoedit. However, this requires that the PostScript program checks for the PostScript level supported by the interpreter and "acts" accordingly. The default language level is 3.


\item[\OptArg{-f}{~"format\Lbr:options\Rbr"}] 
target output format recognized by \Prog{pstoedit}.  Since other format drivers can be loaded dynamically, type  \texttt{pstoedit -help} to get a full list of formats. See  "Available formats and their specific options " below for an explanation of the \oArg{:options} to \Opt{-f} format. If the format option is not given, pstoedit tries to guess the target format  from the suffix of the output filename. However, in a lot of cases, this is not a unique mapping and hence pstoedit demands the \Opt{-f} option.


\item[\oOptArg{-gsregbase}{~GhostScript base registry path}] 
registry path to use as a base path when searching GhostScript interpreter  
This option provides means to specify a registry key under HKLM/Software where to search for GS interpreter key, version and \verb+GS_DLL / GS_LIB+ values. Example: "-gsregbase MyCompany" means  that HKLM/Software/MyCompany/GPL GhostScript would be searched instead of HKLM/Software/GPL GhostScript. 


\end{description}
\subsection{Text and font handling related options}
\begin{description}
\item[\oOptArg{-df}{~font name}] 
 Sometimes fonts embedded in a PostScript program do not have a fontname. For example, this happens in PostScript files generated by \Cmd{dvips}{1}. In such a case \Prog{pstoedit} uses a replacement font. The default for this is Courier. Another font can be specified using the \Opt{-df} option. \Opt{-df Helvetica} causes all unnamed fonts to be replaced by Helvetica. 


\item[\oOpt{-nomaptoisolatin1}] 
Normally \Prog{pstoedit} maps all character codes to the ones defined by the ISO-Latin1 encoding. If you specify \Opt{-nomaptoisolatin1} then the encoding from the input PostScript is passed unchanged to the output. This may result in strange text output but on the other hand may be the only way to get some fonts converted appropriately. Try what fits best to your concrete case. 


\item[\oOptArg{-pngimage}{~filename - for debugging purpose mainly. Write result of processing also to a PNG file.}] 
for debugging purpose mainly. Write result of processing also to a PNG file


\item[\oOpt{-dt}] 
Draw text - Text is drawn as polygons. This might produce a large output file. This option is automatically switched on if the selected output format does not support text, e.g. \Cmd{gnuplot}{1}.


\item[\oOpt{-adt}] 
Automatic Draw text - This option turns on the \Opt{-dt} option selectively for fonts that seem to be no normal text fonts, e.g. Symbol..


\item[\oOpt{-ndt}] 
Never Draw text -  fully disable the heuristics used by pstoedit to decide when to "draw" text instead of showing it as text. This may produce incorrect results, but in some cases it might nevertheless be useful. "Use at own risk". 


\item[\oOpt{-dgbm}] 
experimental - draw also bitmaps generated by fonts/glyphs


\item[\oOpt{-correctdefinefont}] 
Some PostScript files, e.g. such as generated by ChemDraw, use the PostScript definefont operator in way that is incompatible with pstoedit's assumptions. The new font is defined by copying an old font without changing the FontName of the new font. When this option is applied, some "patches" are done after a definefont in order to make it again compatible with pstoedit's assumptions. This option is not enabled per default, since it may break other PostScript file. It is tested only with ChemDraw generated files.


\item[\oOpt{-pti}] 
Precision text - Normally a text string is drawn as it occurs in the input file. However, in some situations, this might produce wrongly positioned characters. This is due to limitiations in most output formats of pstoedit. They cannot represent text with arbitray inter-letter spacing which is easily possible in PDF and PostScript. With \Opt{-pta}, each character of a text string is placed separately. With \Opt{-pti}, this is done only in cases when there is a non zero inter-letter spacing. The downside of "precision text" is a bigger file size and hard to edit text.


\item[\oOpt{-pta}] 
see -pti


\item[\oOptArg{-uchar}{~character}] 
Sometimes pstoedit cannot map a character from the encoding used by the PostScript file to the font encoding of the target format. In this case pstoedit replaces the input character by a special character in order to show all the places that couldn't be mapped correctly. The default for this is a "\#". Using the \Opt{-uchar} option it is possible to specify another character to be used instead. If you want to use a space, use -uchar " ".


\item[\oOpt{-t2fontsast1}] 
Handle type 2 fonts same as type 1. Type 2 fonts sometimes occur as embedded fonts within PDF files. In the default mode, text using such fonts is drawn as polygons since pstoedit assumes that such a font is not available on the users machine. If this option is set, pstoedit assumes that the internal encoding follows the same as for a standard font and generates normal text output. This assumption may not be true in all cases. But it is nearly impossible for pstoedit to verify this assumption - it would have to do a sort of OCR.


\item[\oOpt{-nfr}] 
In normal mode pstoedit replaces bitmap fonts with a font as defined by the \Opt{-df} option. This is done, because most output formats can't handle such fonts. This behavior can be switched off using the \Opt{-nfr} option but then it strongly depends on the application reading the the generated file whether the file is usable and correctly interpreted or not. Any problems are then out of control of pstoedit.


\item[\oOpt{-glyphs}] 
pass glyph names to the output format driver. So far no output format driver really uses the glyph names, so this does not have any effect at the moment. It is a preparation for future work.


\item[\oOpt{-useoldnormalization}] 
Just use this option in case the new heuristic introduced in 3.5 doesn't produce correct results - however, this normalization of font encoding will always be a best-effort approach since there is no real general solution to it with reasonable effort


\item[\oOptArg{-fontmap}{~name of font map file for pstoedit}] 
The font map is a simple text file containing lines in the following format:\\  
 

\verb+document_font_name    target_font_name+ \\
Lines beginning with \verb+%+ are considerd comments \\
 If a font name contains spaces, use the \verb+"font name with spaces"+ notation. 
 
Each font name found in the document is checked against this mapping and if there is a corresponding entry, the new name is used for the output.  

If  the \Opt{-fontmap} option is not specified, \Prog{pstoedit} automatically looks for the file \emph{drivername}.fmp in the installation directory and uses that file as a default fontmap file if available. The installation directory is:  

\begin{itemize} 
 
\item Windows: The same directory where the \Prog{pstoedit} executable is 
    located  
 
\item Unix: \\ 
$<$\emph{The directory where the pstoedit executably is located}$>$\verb+/../lib/+ 
 
\end{itemize} 
 
The mpost.fmp in the misc directory of the pstoedit distibution is a sample map file with mappings from over 5000 PostScript font names to their \TeX equivalents. This is useful because MetaPost is frequently used with \TeX/\LaTeX\ and those programs don't use standard font names. This file and the MetaPost output format driver are provided by Scott Pakin (\Email{scott+ps2ed_AT_pakin.org}).   Another example is wemf.fmp to be used under Windows. See the misc directory of the pstoedit source distribution. 


\end{description}
\subsection{Debug options}
\begin{description}
\item[\oOpt{-dis}] 
Open a display during processing by Ghostscript. Some files only work correctly this way. 


\item[\oOpt{-q}] 
quiet mode - do not write startup message


\item[\oOpt{-nq}] 
No exit from the PostScript interpreter. Normally Ghostscript exits after processing the pstoedit input-file. For debugging it can be useful to avoid this. If you do, you will have to type quit at the \verb+GS>+ prompt to exit from Ghostscript.   


\item[\oOpt{-v}] 
Switch on verbose mode. Some additional information is shown during processing. 


\item[\oOpt{-nb}] 
Since version 3.10 \Prog{pstoedit} uses the \texttt{-dDELAYBIND} option when calling GhostScript. Previously the \texttt{-dNOBIND} option was used instead but that sometimes caused problems if a user's PostScript file overloaded standard PostScript operator with totally new semantic, e.g. lt for lineto instead of the standard meaning of "less than". Using \Opt{-nb} the old style can be activated again in case the \texttt{-dDELAYBIND} gives different results as before. In such a case please also contact the author. 


\item[\oOpt{-ups}] 
write text as plain string instead of hex string in intermediate format - normally useful for trouble shooting and debugging only.


\item[\oOpt{-keep}] 
keep the intermediate files produced by pstoedit - for debug purposes only


\item[\oOpt{-gstest}] 
perform a basic test for the interworking with GhostScript


\end{description}
\subsection{Drawing related options}
\begin{description}
\item[\oOpt{-nc}] 
no curves.  
Normally pstoedit tries to keep curves from the input and transfers them to the output if the output format supports curves. If the output format does not support curves, then pstoedit replaces curves by a series of lines (see also \Opt{-flat} option). However, in some cases the user might wish to have this behavior also for output formats that originally support curves. This can be forced via the \Opt{-nc} option. 


\item[\oOpt{-nsp}] 
normally subpathes are used if the output format support them. This option turns off subpathes.


\item[\oOpt{-mergelines}] 
Some output formats permit the representation of filled polygons with edges that are in a different color than the fill color. Since PostScript does not support this by the standard drawing primitives directly, drawing programs typically generate two objects (the outline and the filled polygon) into the PostScript output. \Prog{pstoedit} is able to recombine these, if they follow each other directly and you specify \Opt{-mergelines}. However, this merging is not supported by all output formats due to restrictions in the target format.


\item[\oOpt{-filledrecttostroke}] 
Rectangles filled with a solid color can be converted to a stroked line with a width that corresponds to the width of the rectangle. This is of primary interest for output formats which do not support filled polygons at all. But it is restricted to rectangles only, i.e. it is not supported for general polygons


\item[\oOpt{-mergetext}] 
In order to produce nice looking text output, programs producing PostScript files often split words into smaller pieces which are then placed individually on adjacent positions. However, such split text is hard to edit later on and hence it is sometime better to recombine these pieces again to form a word (or even sequence of words). For this pstoedit implements some heuristics about what text pieces are to be considered parts of a split word. This is based on the geometrical proximity of the different parts and seems to work quite well so far. But there are certainly cases where this simple heuristic fails. So please check the results carefully.


\item[\oOpt{-ssp}] 
simulate sub paths. 
Several output formats don't support PostScript pathes containing sub pathes, i.e. pathes with intermediate movetos. In the normal case, each subpath is treated as an independent path for such output formats. This can lead to bad looking results. The most common case where this happens is if you use the \Opt{-dt} option and show some text with letters like e, o, or b, i.e. letter that have a "hole". When the \Opt{-ssp} option is set, pstoedit tries to eliminate these problems. However, this option is CPU time intensive! 


\item[\oOptArg{-flat}{~flatness factor}] 
If the output format does not support curves in the way PostScript does or if the \Opt{-nc} option is specified, all curves are approximated by lines. Using the \Opt{-flat} option one can control this approximation. This parameter is directly converted to a PostScript \textbf{setflat} command. Higher numbers, e.g. 10 give rougher, lower numbers, e.g. 0.1 finer approximations.  


\item[\oOpt{-sclip}] 
simulate clipping.  
Most output formats of pstoedit don't have native support for clipping. For that \Prog{pstoedit} offers an option to perform the clipping of the graphics directly without passing the clippath to the output driver. However, this results in curves being replaced by a lot of line segments and thus larger output files. So use this option only if your output looks different from the input due to clipping. In addition, this "simulated clipping" is not exactly the same as defined in PostScript. There might be lines drawn at the double size. Also clipping of text is not supported unless you also use the \Opt{-dt} option.  


\end{description}
\subsection{Input and outfile file arguments}
[ inputfile [outputfile] ] 


If neither an input nor an output file is given as argument, pstoedit works as filter reading from standard input and
writing to standard output. 
The special filename "-" can also be used. It represents standard input if it is the first on the command line and standard output if it is the second. So "pstoedit - output.xxx" reads from standard input and writes to output.xxx


\section{Available formats and their specific options}

\Prog{pstoedit} allows passing individual options to a output format driver. This is done by
appending all options to the format specified after the \Opt{-f} option. The format
specifier and its options must be separated by a colon (:). If more than one
option needs to be passed to the output format driver, the whole argument to \Opt{-f} must be
enclosed within double-quote characters, thus:

\OptArg{-f}{~"format[:option option ...]"} 

To see which options are supported by a specific format, type:
     \textbf{pstoedit -f format:-help}  
     \\ 

The following description of the different formats supported by pstoedit is extracted from the source code of the individual drivers.

\subsubsection{psf - Flattened PostScript (no curves)}
No driver specific options
%%// end of options 
\subsubsection{ps - Simplified PostScript with curves}
No driver specific options
%%// end of options 
\subsubsection{debug - for test purposes}
No driver specific options
%%// end of options 
\subsubsection{dump - for test purposes (same as debug)}
No driver specific options
%%// end of options 
\subsubsection{gs - any device that GhostScript provides - use gs:format, e.g. gs:pdfwrite}
No driver specific options
%%// end of options 
\subsubsection{ps2ai - Adobe Illustrator via ps2ai.ps of GhostScript}
No driver specific options
%%// end of options 
\subsubsection{gmfa - ASCII GNU metafile }
\begin{description}
\item[\oOptArg{plotformat}{~string}] 
plotutil format to generate


\end{description}
%%// end of options 
\subsubsection{gmfb - binary GNU metafile }
\begin{description}
\item[\oOptArg{plotformat}{~string}] 
plotutil format to generate


\end{description}
%%// end of options 
\subsubsection{plot - GNU libplot output types, e.g. plot:type X}
\begin{description}
\item[\oOptArg{plotformat}{~string}] 
plotutil format to generate


\end{description}
%%// end of options 
\subsubsection{plot-cgm - cgm  via GNU libplot}
\begin{description}
\item[\oOptArg{plotformat}{~string}] 
plotutil format to generate


\end{description}
%%// end of options 
\subsubsection{plot-ai - ai   via GNU libplot}
\begin{description}
\item[\oOptArg{plotformat}{~string}] 
plotutil format to generate


\end{description}
%%// end of options 
\subsubsection{plot-svg - svg  via GNU libplot}
\begin{description}
\item[\oOptArg{plotformat}{~string}] 
plotutil format to generate


\end{description}
%%// end of options 
\subsubsection{plot-ps - ps   via GNU libplot}
\begin{description}
\item[\oOptArg{plotformat}{~string}] 
plotutil format to generate


\end{description}
%%// end of options 
\subsubsection{plot-fig - fig  via GNU libplot}
\begin{description}
\item[\oOptArg{plotformat}{~string}] 
plotutil format to generate


\end{description}
%%// end of options 
\subsubsection{plot-pcl - pcl  via GNU libplot}
\begin{description}
\item[\oOptArg{plotformat}{~string}] 
plotutil format to generate


\end{description}
%%// end of options 
\subsubsection{plot-hpgl - hpgl via GNU libplot}
\begin{description}
\item[\oOptArg{plotformat}{~string}] 
plotutil format to generate


\end{description}
%%// end of options 
\subsubsection{plot-tek - tek  via GNU libplot}
\begin{description}
\item[\oOptArg{plotformat}{~string}] 
plotutil format to generate


\end{description}
%%// end of options 
\subsubsection{magick - MAGICK driver}
This driver uses the C++ API of ImageMagick or GraphicsMagick to finally produce different output formats. The output format is determined automatically by Image/GraphicsMagick based on the suffix of the output filename. So an output file test.png will force the creation of an image in PNG format.

No driver specific options
%%// end of options 
\subsubsection{swf - SWF driver: }
\begin{description}
\item[\oOpt{-cubic}] 
cubic ???


\item[\oOpt{-trace}] 
trace ???


\end{description}
%%// end of options 
\subsubsection{svg - scalable vector graphics}
\begin{description}
\item[\oOpt{-localdtd}] 
use local DTD


\item[\oOpt{-standalone}] 
create stand-alone type svg


\item[\oOpt{-withdtd}] 
write DTD


\item[\oOpt{-withgrouping}] 
write also ordinary save/restores as SVG group


\item[\oOpt{-nogroupedpath}] 
do not write a group around pathes


\item[\oOpt{-noviewbox}] 
don't write a view box


\item[\oOpt{-texmode}] 
TeX Mode


\item[\oOpt{-imagetofile}] 
write raster images to separate files instead of embedding them


\item[\oOpt{-notextrendering}] 
do not write textrendering attribute


\item[\oOptArg{-border}{~number}] 
additional border to draw around bare bounding box (in percent of width and height)


\item[\oOptArg{-title}{~string}] 
text to use as title for the generated document


\end{description}
%%// end of options 
\subsubsection{xaml - eXtensible Application Markup Language}
\begin{description}
\item[\oOpt{-localdtd}] 
use local DTD


\item[\oOpt{-standalone}] 
create stand-alone type svg


\item[\oOpt{-withdtd}] 
write DTD


\item[\oOpt{-withgrouping}] 
write also ordinary save/restores as SVG group


\item[\oOpt{-nogroupedpath}] 
do not write a group around pathes


\item[\oOpt{-noviewbox}] 
don't write a view box


\item[\oOpt{-texmode}] 
TeX Mode


\item[\oOpt{-imagetofile}] 
write raster images to separate files instead of embedding them


\item[\oOpt{-notextrendering}] 
do not write textrendering attribute


\item[\oOptArg{-border}{~number}] 
additional border to draw around bare bounding box (in percent of width and height)


\item[\oOptArg{-title}{~string}] 
text to use as title for the generated document


\end{description}
%%// end of options 
\subsubsection{cgmb1 - CGM binary Format (V1)}
No driver specific options
%%// end of options 
\subsubsection{cgmb - CGM binary Format (V3)}
No driver specific options
%%// end of options 
\subsubsection{cgmt - CGM textual Format}
No driver specific options
%%// end of options 
\subsubsection{mif - (Frame)Maker Intermediate Format}
\begin{description}
\item[\oOpt{-nopage}] 
do not add a separate Page entry


\end{description}
%%// end of options 
\subsubsection{rtf - RTF Format}
No driver specific options
%%// end of options 
\subsubsection{wemf - Wogls version of EMF }
\begin{description}
\item[\oOpt{-df}] 
write info about font processing


\item[\oOpt{-dumpfontmap}] 
write info about font mapping


\item[\oOpt{-size:psbbox}] 
use the bounding box as calculated by the PostScript frontent as size


\item[\oOpt{-size:fullpage}] 
set the size to the size of a full page


\item[\oOpt{-size:automatic}] 
let windows calculate the bounding box (default)


\item[\oOpt{-keepimages}] 
debug option - keep the embedded bitmaps as external files 


\item[\oOpt{-useoldpolydraw}] 
do not use Windows PolyDraw but an emulation of it - sometimes needed for certain programs reading the EMF files


\item[\oOpt{-OO}] 
generate OpenOffice compatible EMF file


\end{description}
%%// end of options 
\subsubsection{wemfc - Wogls version of EMF with experimental clip support}
\begin{description}
\item[\oOpt{-df}] 
write info about font processing


\item[\oOpt{-dumpfontmap}] 
write info about font mapping


\item[\oOpt{-size:psbbox}] 
use the bounding box as calculated by the PostScript frontent as size


\item[\oOpt{-size:fullpage}] 
set the size to the size of a full page


\item[\oOpt{-size:automatic}] 
let windows calculate the bounding box (default)


\item[\oOpt{-keepimages}] 
debug option - keep the embedded bitmaps as external files 


\item[\oOpt{-useoldpolydraw}] 
do not use Windows PolyDraw but an emulation of it - sometimes needed for certain programs reading the EMF files


\item[\oOpt{-OO}] 
generate OpenOffice compatible EMF file


\end{description}
%%// end of options 
\subsubsection{wemfnss - Wogls version of EMF - no subpathes }
\begin{description}
\item[\oOpt{-df}] 
write info about font processing


\item[\oOpt{-dumpfontmap}] 
write info about font mapping


\item[\oOpt{-size:psbbox}] 
use the bounding box as calculated by the PostScript frontent as size


\item[\oOpt{-size:fullpage}] 
set the size to the size of a full page


\item[\oOpt{-size:automatic}] 
let windows calculate the bounding box (default)


\item[\oOpt{-keepimages}] 
debug option - keep the embedded bitmaps as external files 


\item[\oOpt{-useoldpolydraw}] 
do not use Windows PolyDraw but an emulation of it - sometimes needed for certain programs reading the EMF files


\item[\oOpt{-OO}] 
generate OpenOffice compatible EMF file


\end{description}
%%// end of options 
\subsubsection{hpgl - HPGL code}
\begin{description}
\item[\oOpt{-penplotter}] 
plotter is pen plotter (i.e. no support for specific line widths)


\item[\oOpt{-pencolorsfromfile}] 
read pen colors from file drvhpgl.pencolors in pstoedit data directory


\item[\oOptArg{-pencolors}{~number}] 
maximum number of pen colors to be used by pstoedit (default 0) - 


\item[\oOptArg{-filltype}{~string}] 
select fill type e.g. FT 1


\item[\oOpt{-hpgl2}] 
Use HPGL/2 instead of HPGL/1


\item[\oOpt{-rot90}] 
rotate hpgl by 90 degrees


\item[\oOpt{-rot180}] 
rotate hpgl by 180 degrees


\item[\oOpt{-rot270}] 
rotate hpgl by 270 degrees


\end{description}
%%// end of options 
\subsubsection{pcl - PCL code}
\begin{description}
\item[\oOpt{-penplotter}] 
plotter is pen plotter (i.e. no support for specific line widths)


\item[\oOpt{-pencolorsfromfile}] 
read pen colors from file drvhpgl.pencolors in pstoedit data directory


\item[\oOptArg{-pencolors}{~number}] 
maximum number of pen colors to be used by pstoedit (default 0) - 


\item[\oOptArg{-filltype}{~string}] 
select fill type e.g. FT 1


\item[\oOpt{-hpgl2}] 
Use HPGL/2 instead of HPGL/1


\item[\oOpt{-rot90}] 
rotate hpgl by 90 degrees


\item[\oOpt{-rot180}] 
rotate hpgl by 180 degrees


\item[\oOpt{-rot270}] 
rotate hpgl by 270 degrees


\end{description}
%%// end of options 
\subsubsection{pic - PIC format for troff et.al.}
\begin{description}
\item[\oOpt{-troff}] 
troff mode (default is groff)


\item[\oOpt{-landscape}] 
landscape output


\item[\oOpt{-portrait}] 
portrait output


\item[\oOpt{-keepfont}] 
print unrecognized literally


\item[\oOpt{-text}] 
try not to make pictures from running text


\item[\oOpt{-debug}] 
enable debug output


\end{description}
%%// end of options 
\subsubsection{asy - Asymptote Format}
No driver specific options
%%// end of options 
\subsubsection{cairo - cairo driver}
generates compilable c code for rendering with cairo

\begin{description}
\item[\oOpt{-pango}] 
use pango for font rendering


\item[\oOptArg{-funcname}{~string}] 
sets the base name for the generated functions and variables.  e.g. myfig


\item[\oOptArg{-header}{~string}] 
sets the output file name for the generated C header file.  e.g. myfig.h


\end{description}
%%// end of options 
\subsubsection{cfdg - Context Free Design Grammar}
Context Free Design Grammar, usable by Context Free Art (http://www.contextfreeart.org/)

No driver specific options
%%// end of options 
\subsubsection{dxf - CAD exchange format}
\begin{description}
\item[\oOpt{-polyaslines}] 
use LINE instead of POLYLINE in DXF


\item[\oOpt{-mm}] 
use mm coordinates instead of points in DXF (mm=pt/72*25.4)


\item[\oOpt{-ctl}] 
map colors to layers


\item[\oOpt{-splineaspolyline}] 
approximate splines with PolyLines (only for -f dxf\_s)


\item[\oOpt{-splineasnurb}] 
experimental (only for -f dxf\_s)


\item[\oOpt{-splineasbspline}] 
experimental (only for -f dxf\_s)


\item[\oOpt{-splineassinglespline}] 
experimental (only for -f dxf\_s)


\item[\oOpt{-splineasmultispline}] 
experimental (only for -f dxf\_s)


\item[\oOpt{-splineasbezier}] 
use Bezier splines in DXF format (only for -f dxf\_s)


\item[\oOptArg{-splineprecision}{~number}] 
number of samples to take from spline curve when doing approximation with -splineaspolyline or -splineasmultispline - should be $>=$ 2 (default 5)


\item[\oOpt{-dumplayernames}] 
dump all layer names found to standard output


\item[\oOptArg{-layers}{~string}] 
layers to be shown (comma separated list of layer names, no space)


\item[\oOptArg{-layerfilter}{~string}] 
layers to be hidden (comma separated list of layer names, no space)


\end{description}
%%// end of options 
\subsubsection{dxf\_s - CAD exchange format with splines}
\begin{description}
\item[\oOpt{-polyaslines}] 
use LINE instead of POLYLINE in DXF


\item[\oOpt{-mm}] 
use mm coordinates instead of points in DXF (mm=pt/72*25.4)


\item[\oOpt{-ctl}] 
map colors to layers


\item[\oOpt{-splineaspolyline}] 
approximate splines with PolyLines (only for -f dxf\_s)


\item[\oOpt{-splineasnurb}] 
experimental (only for -f dxf\_s)


\item[\oOpt{-splineasbspline}] 
experimental (only for -f dxf\_s)


\item[\oOpt{-splineassinglespline}] 
experimental (only for -f dxf\_s)


\item[\oOpt{-splineasmultispline}] 
experimental (only for -f dxf\_s)


\item[\oOpt{-splineasbezier}] 
use Bezier splines in DXF format (only for -f dxf\_s)


\item[\oOptArg{-splineprecision}{~number}] 
number of samples to take from spline curve when doing approximation with -splineaspolyline or -splineasmultispline - should be $>=$ 2 (default 5)


\item[\oOpt{-dumplayernames}] 
dump all layer names found to standard output


\item[\oOptArg{-layers}{~string}] 
layers to be shown (comma separated list of layer names, no space)


\item[\oOptArg{-layerfilter}{~string}] 
layers to be hidden (comma separated list of layer names, no space)


\end{description}
%%// end of options 
\subsubsection{fig - .fig format for xfig}
The xfig format driver supports special fontnames, which may be produced by using a fontmap file. The following types of names are supported : \\  
\begin{verbatim}
General notation: 
"Postscript Font Name" ((LaTeX|PostScript|empty)(::special)::)XFigFontName
 
Examples:

Helvetica LaTeX::SansSerif
Courier LaTeX::special::Typewriter
GillSans "AvantGarde Demi"
Albertus PostScript::special::"New Century Schoolbook Italic" 
Symbol ::special::Symbol (same as Postscript::special::Symbol)
\end{verbatim}
See also the file examplefigmap.fmp in the misc directory of the pstoedit source distribution for an example font map file for xfig. Please note that the Fontname has to be among those supported by xfig. See - \URL{http://www.xfig.org/userman/fig-format.html} for a list of legal font names

\begin{description}
\item[\oOptArg{-startdepth}{~number}] 
Set the initial depth (default 999)


\item[\oOpt{-metric}] 
Switch to centimeter display (default inches)


\item[\oOpt{-usecorrectfontsize}] 
don't scale fonts for xfig. Use this if you also use this option with xfig


\item[\oOptArg{-depth}{~number}] 
Set the page depth in inches (default 11)


\end{description}
%%// end of options 
\subsubsection{xfig - .fig format for xfig}
See fig format for more details.

\begin{description}
\item[\oOptArg{-startdepth}{~number}] 
Set the initial depth (default 999)


\item[\oOpt{-metric}] 
Switch to centimeter display (default inches)


\item[\oOpt{-usecorrectfontsize}] 
don't scale fonts for xfig. Use this if you also use this option with xfig


\item[\oOptArg{-depth}{~number}] 
Set the page depth in inches (default 11)


\end{description}
%%// end of options 
\subsubsection{tfig - .fig format for xfig}
Test only

\begin{description}
\item[\oOptArg{-startdepth}{~number}] 
Set the initial depth (default 999)


\item[\oOpt{-metric}] 
Switch to centimeter display (default inches)


\item[\oOpt{-usecorrectfontsize}] 
don't scale fonts for xfig. Use this if you also use this option with xfig


\item[\oOptArg{-depth}{~number}] 
Set the page depth in inches (default 11)


\end{description}
%%// end of options 
\subsubsection{gcode - emc2 gcode format}
See also:  \URL{http://linuxcnc.org/} 

No driver specific options
%%// end of options 
\subsubsection{gnuplot - gnuplot format}
No driver specific options
%%// end of options 
\subsubsection{gschem - gschem format}
See also:  \URL{http://www.geda.seul.org/tools/gschem/} 

No driver specific options
%%// end of options 
\subsubsection{idraw - Interviews draw format (EPS)}
No driver specific options
%%// end of options 
\subsubsection{java1 - java 1 applet source code}
\begin{description}
\item[\oOptArg{java class name}{~string}] 
name of java class to generate


\end{description}
%%// end of options 
\subsubsection{java2 - java 2 source code}
\begin{description}
\item[\oOptArg{java class name}{~string}] 
name of java class to generate


\end{description}
%%// end of options 
\subsubsection{kil - .kil format for Kontour}
No driver specific options
%%// end of options 
\subsubsection{latex2e - LaTeX2e picture format}
\begin{description}
\item[\oOpt{-integers}] 
round all coordinates to the nearest integer


\end{description}
%%// end of options 
\subsubsection{lwo - LightWave 3D Object Format}
No driver specific options
%%// end of options 
\subsubsection{mma - Mathematica Graphics}
\begin{description}
\item[\oOpt{-eofillfills}] 
Filling is used for eofill (default is not to fill)


\end{description}
%%// end of options 
\subsubsection{mpost - MetaPost Format}
No driver specific options
%%// end of options 
\subsubsection{noixml - Nemetschek NOI XML format}
Nemetschek Object Interface XML format

\begin{description}
\item[\oOptArg{-r}{~string}] 
Allplan resource file


\item[\oOptArg{-bsl}{~number}] 
Bezier Split Level (default 3)


\end{description}
%%// end of options 
\subsubsection{pcbi - engrave data - insulate/PCB format}
See \URL{http://home.vr-web.de/\Tilde hans-juergen-jahn/software/devpcb.html} for more details.

No driver specific options
%%// end of options 
\subsubsection{pcb - pcb format}
See also: \URL{http://pcb.sourceforge.net} and \URL{http://www.penguin.cz/\Tilde utx/pstoedit-pcb/} 

\begin{description}
\item[\oOptArg{-grid}{~missing arg name}] 
attempt to snap relevant output to grid (mils) and put failed objects to a different layer


\item[\oOptArg{-snapdist}{~missing arg name}] 
grid snap distance ratio (0 < snapdist <= 0.5, default 0.1)


\item[\oOptArg{-tshiftx}{~missing arg name}] 
additional x shift measured in target units (mils)


\item[\oOptArg{-tshifty}{~missing arg name}] 
additional y shift measured in target units (mils)


\item[\oOptArg{-grid}{~missing arg name}] 
attempt to snap relevant output to grid (mils) and put failed objects to a different layer


\item[\oOpt{-mm}] 
Switch to metric units (mm)


\item[\oOpt{-stdnames}] 
use standard layer names instead of descriptive names


\item[\oOpt{-forcepoly}] 
force all objects to be interpreted as polygons


\end{description}
%%// end of options 
\subsubsection{pcbfill - pcb format with fills}
See also: \URL{http://pcb.sourceforge.net} 

No driver specific options
%%// end of options 
\subsubsection{pdf - Adobe's Portable Document Format}
No driver specific options
%%// end of options 
\subsubsection{rib - RenderMan Interface Bytestream}
No driver specific options
%%// end of options 
\subsubsection{rpl - Real3D Programming Language Format}
No driver specific options
%%// end of options 
\subsubsection{sample - sample driver: if you don't want to see this, uncomment the corresponding line in makefile and make again}
this is a long description for the sample driver

\begin{description}
\item[\oOptArg{-sampleoption}{~integer}] 
just an example


\end{description}
%%// end of options 
\subsubsection{sk - Sketch Format}
No driver specific options
%%// end of options 
\subsubsection{svm - StarView/OpenOffice.org metafile}
StarView/OpenOffice.org metafile, readable from OpenOffice.org 1.0/StarOffice 6.0 and above.

\begin{description}
\item[\oOpt{-m}] 
map to Arial


\item[\oOpt{-nf}] 
emulate narrow fonts


\end{description}
%%// end of options 
\subsubsection{text - text in different forms }
\begin{description}
\item[\oOptArg{-height}{~number}] 
page height in terms of characters


\item[\oOptArg{-width}{~number}] 
page width in terms of characters


\item[\oOpt{-dump}] 
dump text pieces


\end{description}
%%// end of options 
\subsubsection{tgif - Tgif .obj format}
\begin{description}
\item[\oOpt{-ta}] 
text as attribute


\end{description}
%%// end of options 
\subsubsection{tk - tk and/or tk applet source code}
\begin{description}
\item[\oOpt{-R}] 
swap HW


\item[\oOpt{-I}] 
no impress


\item[\oOptArg{-n}{~string}] 
tagnames


\end{description}
%%// end of options 
\subsubsection{vtk - VTK driver: if you don't want to see this, uncomment the corresponding line in makefile and make again}
this is a long description for the VTKe driver

\begin{description}
\item[\oOptArg{-VTKeoption}{~integer}] 
just an example


\end{description}
%%// end of options 
\subsubsection{wmf - Windows metafile}
\begin{description}
\item[\oOpt{-m}] 
map to Arial


\item[\oOpt{-nf}] 
emulate narrow fonts


\item[\oOpt{-drawbb}] 
draw bounding box


\item[\oOpt{-p}] 
prune line ends


\item[\oOpt{-nfw}] 
Newer versions of Windows (2000, XP, Vista) will not accept WMF/EMF files generated when this option is set and the input contains Text. But if this option is not set, then the WMF/EMF driver will estimate interletter spacing of text using a very coarse heuristic. This may result in ugly looking output. On the other hand, OpenOffice can still read EMF/WMF files where pstoedit delegates the calculation of the inter letter spacing to the program reading the WMF/EMF file. So if the generated WMF/EMF file shall never be processed under Windows, use this option. If WMF/EMF files with high precision text need to be generated under *nix the only option is to use the -pta option of pstoedit. However that causes every text to be split into single characters which makes the text hard to edit afterwards. Hence the -nfw options provides a sort of compromise between portability and nice to edit but still nice looking text. Again - this option has no meaning when pstoedit is executed under Windows anyway. In that case the output is portable but nevertheless not split and still looks fine.


\item[\oOpt{-winbb}] 
let the Windows API calculate the Bounding Box (Windows only)


\item[\oOpt{-OO}] 
generate OpenOffice compatible EMF file


\end{description}
%%// end of options 
\subsubsection{emf - Enhanced Windows metafile}
\begin{description}
\item[\oOpt{-m}] 
map to Arial


\item[\oOpt{-nf}] 
emulate narrow fonts


\item[\oOpt{-drawbb}] 
draw bounding box


\item[\oOpt{-p}] 
prune line ends


\item[\oOpt{-nfw}] 
Newer versions of Windows (2000, XP, Vista) will not accept WMF/EMF files generated when this option is set and the input contains Text. But if this option is not set, then the WMF/EMF driver will estimate interletter spacing of text using a very coarse heuristic. This may result in ugly looking output. On the other hand, OpenOffice can still read EMF/WMF files where pstoedit delegates the calculation of the inter letter spacing to the program reading the WMF/EMF file. So if the generated WMF/EMF file shall never be processed under Windows, use this option. If WMF/EMF files with high precision text need to be generated under *nix the only option is to use the -pta option of pstoedit. However that causes every text to be split into single characters which makes the text hard to edit afterwards. Hence the -nfw options provides a sort of compromise between portability and nice to edit but still nice looking text. Again - this option has no meaning when pstoedit is executed under Windows anyway. In that case the output is portable but nevertheless not split and still looks fine.


\item[\oOpt{-winbb}] 
let the Windows API calculate the Bounding Box (Windows only)


\item[\oOpt{-OO}] 
generate OpenOffice compatible EMF file


\end{description}
%%// end of options 

\section{NOTES}


  \subsection{autotrace}

	pstoedit cooperates with autotrace. Autotrace can now produce a dump file 
	for further processing by pstoedit using the \Opt{-bo} (backend only) option. 
	Autotrace is a program written by a group around Martin Weber and can be 
	found at \URL{http://sourceforge.net/projects/autotrace/}.

  \subsection{Ps2ai}

    The ps2ai output format driver is not a native pstoedit output format driver. It does not use the
    pstoedit postcript flattener, instead it uses the PostScript program
    ps2ai.ps which is installed in the GhostScript distribution directory. It
    is included  to provide the same "look-and-feel" for the conversion to AI.
    The additional benefit is that this conversion is now available also via
    the "convert-to-vector" menu of Gsview. However, lot's of files don't
    convert nicely or at all using ps2ai.ps. So a native pstoedit driver would
    be much better. Anyone out there to take this? The AI format is usable for
    example by Mayura Draw (\URL{http://www.mayura.com}). Also a driver to the
    Mayura native format would be nice. 

    An alternative to the ps2ai based driver is available via the -f plot:ai format if the libplot(ter) is installed.

    You should use a version of GhostScript greater than or equal to 6.00 for using the ps2ai output format driver.


  \subsection{MetaPost}

    Note that, as far as Scott knows, MetaPost does not support PostScript's
    eofill. The metapost output format driver just converts eofill to fill, and issues a warning if
    verbose is set. Fortunately, very few PostScript programs rely on the
    even-odd fill rule, even though many specify it.

    For more on MetaPost see: 
    
    \URL{http://cm.bell-labs.com/who/hobby/MetaPost.html}

  \subsection{Context Free - CFDG} 
	The driver for the CFDG format (drvcfdg) defines
	one shape per page of PostScript, but only the first shape is actually
	rendered (unless the user edits the generated CFDG code, of course).
	CFDG doesn't support multi-page output, so this probably a reasonable thing to do.
 
	For more on Context Free see:
	\URL{http://www.contextfreeart.org/}

  \subsection{LaTeX2e} 

    \begin{itemize}
   \item LaTeX2e's picture environment is not very powerful.  As a result, many
     elementary PostScript constructs are ignored -- fills, line
     thicknesses (besides "thick" and "thin"), and dash patterns, to name a
     few.  Furthermore, complex pictures may overrun TeX's memory capacity.

   \item Some PostScript constructs are not supported directly by "picture",
     but can be handled by external packages.  If a figure uses color, the
     top-level document will need to do a \verb+"\usepackage{color}"+.  And if a
     figure contains rotated text, the top-level document will need to do a
     \verb+"\usepackage{rotating}"+.

   \item All lengths, coordinates, and font sizes output by the output format driver are in
     terms of \verb+\unitlength+, so scaling a figure is simply a matter of doing
     a \verb+"\setlength{\unitlength}{...}"+.

   \item The output format driver currently supports one output format driver specific option,
     "integers", which rounds all lengths, coordinates, and font sizes to
     the nearest integer.  This makes hand-editing the picture a little
     nicer.

   \item Why is this output format driver useful?  One answer is portability; any LaTeX2e
     system can handle the picture environment, even if it can't handle
     PostScript graphics.  (pdfLaTeX comes to mind here.)  A second answer
     is that pictures can be edited easily to contain any arbitrary LaTeX2e
     code.  For instance, the text in a figure can be modified to contain
     complex mathematics, non-Latin alphabets, bibliographic citations, or
     -- the real reason Scott wrote the LaTeX2e output format driver -- hyperlinks to the
     surrounding document (with help from the hyperref package).
   \end{itemize}


  \subsection{Creating a new output format driver}

    To implement a new output format driver you can start from \File{drvsampl.cpp} and
    \File{drvsampl.h}. See also comments in \File{drvbase.h} and
    \File{drvfuncs.h} for an explanation of methods that should be implemented
    for a new output format driver.


\section{ENVIRONMENT VARIABLES}

A default PostScript interpreter to be called by pstoedit is specified at
compile time. You can overwrite the default by setting the GS environment
variable to the name of a suitable PostScript interpreter.

You can check which name of a PostScript interpreter was compiled into
pstoedit using: \textbf{pstoedit} \Opt{-help -v}.

See the GhostScript manual for descriptions of environment variables used by
Ghostscript most importantly \verb+GS_FONTPATH+ and \verb+GS_LIB+; other
environment variables also affect output to display, print, and additional
filtering and processing. See the related documentation.

\Prog{pstoedit} allocates temporary files using the function \Cmd{tempnam}{3}.
Thus the location for temporary files might be controllable by other
environment variables used by this function. See the \Cmd{tempnam}{3} manpage
for descriptions of environment variables used. On UNIX like system this is
probably the \verb+TMPDIR+ variable, on DOS/WINDOWS either \verb+TMP+ or
\verb+TEMP+.

\section{TROUBLE SHOOTING}

If you have problems with \Prog{pstoedit} first try whether Ghostscript
successfully displays your file. If yes, then try 
\textbf{pstoedit} \Opt{-f ps} \Arg{infile.ps} \Arg{testfile.ps} 
and check whether \Arg{testfile.ps} still displays correctly using
Ghostscript. If this file doesn't look correctly then there seems to be a
problem with \Prog{pstoedit}'s PostScript frontend. If this file looks good
but the output for a specific format is wrong, the problem is probably in
the output format driver for the specific format. In either case send bug fixes and
reports to the author.

A common problem with PostScript files is that the PostScript file redefines
one of the standard PostScript operators inconsistently. There is no effect
of this if you just print the file since the original PostScript "program"
uses these new operator in the new meaning and does not use the original
ones anymoew. However, when run under the control of pstoedit, these
operators are expected to work with the original semantics.

So far I've seen redefinitions for:

\begin{itemize}

   \item lt - "less-then" to mean "draw a line to"
   \item string - "create a string object" to mean "draw a string"
   \item length - "get the length of e.g. a string" to a "float constant"
   
\end{itemize}

I've included work-arounds for the ones mentioned above, but some others
could show up in addition to those.


\section{RESTRICTIONS}

\begin{itemize}
\item Non-standard fonts (e.g. \TeX bitmap fonts) are mapped to a default font which
can be changed using the \Opt{-df} option. \Prog{pstoedit} chooses the size of
the replacement font such that the width of the string in the original font is
the same as in the replacement font. This is done for each text fragment
displayed. Special character encoding support is limited in this case. If a
character cannot be mapped into the target format, pstoedit displays a '\#'
instead. See also the -uchar option.

\item pstoedit supports bitmap graphics only for some output format drivers.

\item Some output format drivers, e.g. the Gnuplot output format driver or the 3D output format driver (rpl, lwo, rib) do not support text.

\item For most output format drivers pstoedit does not support clipping (mainly due to limitations in the target format). You can try to use the
\Opt{-sclip} option to simulate clipping. However, this doesn't work in all cases
as expected.

\item Special note about the Java output format drivers (java1 and java2).
The java output format drivers generate a java source file that needs other files in
order to be compiled and usable. These other files are Java classes (one
applet and support classes) that allow to step through the individual pages
of a converted PostScript document. This applet can easily be activated from
a html-document. See the \File{contrib/java/java1/readme_java1.txt} or 
\File{contrib/java/java2/readme_java2.htm} file for more details.
 
\end{itemize}

\section{FAQs}

\begin{enumerate}
\item Why do letters like O or B get strange if converted to tgif/xfig
using the \Opt{-dt} option?

This is because most output format drivers don't support composite paths with
intermediate gaps (moveto's) and second don't support very well the (eo)fill
operators of PostScript (winding rule). For such objects \Prog{pstoedit} breaks
them into smaller objects whenever such a gap is found. This results in the
"hole" beeing filled with black color instead of beeing transparent. Since
version 3.11 you can try the \Opt{-ssp} option in combination with the xfig
output format driver.


\item Why does pstoedit produce ugly results from PostScript files generated
by dvips?

TeX documents usually use bitmap fonts. Such fonts cannot be used as native
font in other format. So pstoedit replaces the TeX font with another native
font. Of course, the replacement font will in most cases produce another
look, especially if mathematical symbols are used.
Try to use PostScript fonts instead of the bitmap fonts when generating a PostScript file from TeX or LaTeX.


\end{enumerate}

\section{AUTHOR}

Wolfgang Glunz, \Email{wglunz35_AT_pstoedit.net}, \URL{http://de.linkedin.com/in/wolfgangglunz}


\section{CANONICAL ARCHIVE SITE}

\URL{http://www.pstoedit.net/pstoedit/}

At this site you also find more information about \Prog{pstoedit} and related
programs and hints how to subscribe to a mailing list in order to get informed
about new releases and bug-fixes.

If you like pstoedit - please express so also at Facebook \URL{http://www.facebook.com/pages/pstoedit/260606183958062}.


\section{ACKNOWLEDGEMENTS}

\begin{itemize}\setlength{\itemsep}{0cm}

  \item Klaus Steinberger \Email{Klaus.Steinberger_AT_physik.uni-muenchen.de}
     wrote the initial version of this manpage.

  \item Lar Kaufman revised the increasingly complex
     command syntax diagrams and updated the structure and content of this
     manpage following release 2.5. 

  \item David B. Rosen \Email{rosen_AT_unr.edu} provided ideas and some PostScript
     code from his ps2aplot program.

  \item Ian MacPhedran \Email{Ian_MacPhedran_AT_engr.USask.CA} provided the xfig
     output format driver.

  \item Carsten Hammer \Email{chammer_AT_hermes.hrz.uni-bielefeld.de} provided the
     gnuplot output format driver and the initial DXF output format driver.

  \item Christoph Jaeschke provided the OS/2 metafile (MET) output format driver. 
  Thomas Hoffmann \Email{thoffman_AT_zappa.sax.de} did some further updates on the OS/2 part.

  \item Jens Weber \Email{rz47b7_AT_PostAG.DE} provided the Windows metafile (WMF)
     output format driver, and a graphical user interface (GUI).

  \item G. Edward Johnson \Email{lorax_AT_nist.gov} provided the CGM Draw library
     used in the CGM output format driver.

  \item Gerhard Kircher \Email{kircher_AT_edvz.tuwien.ac.at} provided some bug
     fixes.

  \item Bill Cheng \Email{bill.cheng_AT_acm.org} provided help with the tgif
     format and some changes to tgif to make the output format driver easier to implement.
     \URL{http://bourbon.usc.edu:8001/}

  \item Reini Urban \Email{rurban_AT_sbox.tu-graz.ac.at} provided input for the
     extended DXF output format driver.(\URL{http://autocad.xarch.at/})

  \item Glenn M. Lewis \Email{glenn_AT_gmlewis.com} provided RenderMan (RIB),
     Real3D (RPL), and LightWave 3D (LWO) output format drivers.
     (\URL{http://www.gmlewis.com/})

  \item Piet van Oostrum \Email{piet_AT_cs.ruu.nl} made several bug fixes.
  
  \item Lutz Vieweg \Email{lkv_AT_mania.robin.de} provided several bug fixes and
     suggestions for improvements.
     
  \item Derek B. Noonburg \Email{derekn_AT_vw.ece.cmu.edu} and Rainer Dorsch
     \Email{rd_AT_berlepsch.wohnheim.uni-ulm.de} isolated and resolved a
     Linux-specific core dump problem.

  \item Rob Warner \Email{rcw2_AT_ukc.ac.uk} made pstoedit compile under RiscOS.
  
  \item Patrick Gosling \Email{jpmg_AT_eng.cam.ac.uk} made some suggestions
     regarding the usage of pstoedit in Ghostscript's SAFER mode.

  \item Scott Pakin \Email{scott+ps2ed_AT_pakin.org} for the Idraw output format driver and the 
	autoconf support.

  \item Peter Katzmann \Email{p.katzmann_AT_thiesen.com} for the HPGL output format driver.
  
  \item Chris Cox \Email{ccox_AT_airmail.net} contributed the Tcl/Tk output format driver.
  
  \item Thorsten Behrens \Email{Thorsten_Behrens_AT_public.uni-hamburg.de} and
     Bjoern Petersen for reworking the WMF output format driver.

  \item Leszek Piotrowicz \Email{leszek_AT_sopot.rodan.pl} implemented the image
     support for the xfig driver and a JAVA based GUI.

  \item Egil Kvaleberg \Email{egil_AT_kvaleberg.no} contributed the pic output format driver.
  
  \item Kai-Uwe Sattler \Email{kus_AT_iti.cs.uni-magdeburg.de} implemented the
     output format driver for Kontour.

  \item Scott Pakin,  \Email{scott+ps2ed_AT_pakin.org}   provided the MetaPost and LaTeX2e output format driver.
  
  \item Burkhard Plaum \Email{plaum_AT_IPF.Uni-Stuttgart.de} added support for
     complex filled paths for the xfig output format driver.

  \item Bernhard Herzog \Email{herzog_AT_online.de} contributed the output format driver for
     sketch ( \URL{http://www.skencil.org/} )

  \item Rolf Niepraschk (\Email{niepraschk_AT_ptb.de}) converted the HTML man page
     to LaTeX. This allows to generate the UNIX style and the HTML manual from this
     base format.

  \item Several others sent smaller bug fixed and bug reports. Sorry if I don't
     mention them all here.

  \item Gisbert W. Selke (\Email{gisbert_AT_tapirsoft.de}) for the Java 2 output format driver.
     
  \item Robert S. Maier (\Email{rsm_AT_math.arizona.edu}) for many improvements on
	the libplot output format driver and for libplot itself.
  \item The authors of pstotext (\Email{mcjones_AT_pa.dec.com} and \Email{birrell_AT_pa.dec.com}) 
	for giving me the permission to use their simple PostScript code for 
	performing rotation.
  \item  Daniel Gehriger \Email{gehriger_AT_linkcad.com} for his help concerning the handling of Splines in the DXF format. 
  \item Allen Barnett \Email{libemf_AT_lignumcomputing.com} for his work on the libEMF which allows to create WMF/EMF files under *nix systems.
  \item Dave \Email{dave_AT_opaque.net} for providing the libming which is a multiplatform library for generating SWF files.
  \item Masatake Yamoto for the introduction of autoconf, automake and libtool into pstoedit
  \item Bob Friesenhahn for his help and the building of the Magick++ API to ImageMagick.
  \item But most important: Peter Deutsch \Email{ghost_AT_aladdin.com} and Russell
     Lang \Email{gsview_AT_ghostgum.com.au} for their help and answers regarding
     GhostScript and gsview.

\end{itemize}

\section{LEGAL NOTICES}

Trademarks mentioned are the property of their respective owners.

Some code incorporated in the pstoedit package is subject to copyright or
other intellectual property rights or restrictions including attribution
rights. See the notes in individual files.

\Prog{pstoedit} is controlled under the Free Software Foundation GNU Public
License (GPL). However, this does not apply to importps and the additional
plugins.

Aladdin Ghostscript is a redistributable software package with copyright
restrictions controlled by Aladdin Software.

\Prog{pstoedit} has no other relation to Ghostscript besides calling it in a
subprocess.

The authors, contributors, and distributors of pstoedit are not responsible
for its use for any purpose, or for the results generated thereby.

Restrictions such as the foregoing may apply in other countries according to
international conventions and agreements.


\LatexManEnd

\end{document}
